\documentclass{beamer}
\title{MSBD 5001 Tutorial week 8}
\usetheme{Madrid}
\author{Jiacheng Xia}
\date{Oct. 24, 2018}
\begin{document}
\maketitle
	\begin{frame}{Short Comment on Midterm}
	The midterm exam was not computationally-intensive. The grading is also not strict:
	
	\begin{block}{Example}
	If we have a 5-point question, you can get 
		\begin{itemize}	
		\item 4-5 if you get almost correct answer (with reasonable steps and explain)
		\item 2-3 if you get an rough idea of what the question is about
		\item 0-1 if you don't seem to understand the question
 		\end{itemize}
	\end{block}
	
	\begin{alertblock}{}
	The midterm makes 40\% of final grade, work harder on your project if you feel you didn't get good score.  	
	\end{alertblock}
	\end{frame}
	
	\begin{frame}{Individual Project}
	As stated, a Kaggle in-class competition. The link is posted to Canvas
	\begin{block}{Problem description}
	Our problem is related to analyzing the performance of computer programs: We randomly generated datasets and models to train classification models. Your goal is to predict how long the program takes (testing data).
	\end{block}
	
	\begin{alertblock}{Notice}
		Please don't try to replay the programs on your computer. We generated the data using a powerful cluster, so the results will be absolutely different. 
	\end{alertblock}
	\end{frame}
	
	\begin{frame}{Rules}
	We will create a survey on Canvas. Please fill in the Kaggle account before next lecture. Un-registered players are not eligible for bonus and grading. 
	
	\begin{exampleblock}{Metrics}
		The evaluation metric is the Mean Squared Error between your results and the answer.  
	\end{exampleblock}
	
	\begin{exampleblock}{Ranking}
		Smaller error leads to high ranking, graded and generated by Kaggle.	
	\end{exampleblock}
	\end{frame}
	
	\begin{frame}{Rules Cont'd}
	You are allowed to submit 1 time every day. 
	
	\begin{block}{Leaderboard}
	There are two leaderboards:
	\begin{itemize}
		\item Public leaderboard: Based on a fraction of testing set. Will be see by everyone.
		\item Private leaderboard: Based on the rest of testing set. Available at the end of competition.
	\end{itemize}	
	
	\end{block}
	
	\begin{alertblock}{Final Ranking}
	By the end of competition you can select a solution for final grading, by default it's the one with best public leaderboard score. This is graded with private leaderboard.	
	\end{alertblock}
	\end{frame}

	\begin{frame}{Updated Grading Scheme}
	Please notice the updated grading schemes, your grade consists of three parts:
	\begin{itemize}
		\item Valid submission. You have a valid score on Kaggle AND you post your code on Github for validation. Kaggle posts are not required.
		\item Ranking. The leader gets 10 points, 1st runner-up: 9.9 and so on. The last few competitors is 0 since we have more than 100 students.
		\item Bonus. The leader on public leaderboard every week gets 1 point, can be repeated. The top-3 on private leaderboard gets 5,3,2 respectively. Private leaderboard winners needs to do a in-class presentation of your method in the last lecture (or a short interview).
	\end{itemize}
	\end{frame}
	
	\begin{frame}{Questions}
	This is the first time we introduce this competition it may be buggy. We welcome your comments. Please post on Canvas so everyone can see it. 
	\end{frame}


\end{document}